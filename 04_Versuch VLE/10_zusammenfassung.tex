\section{Zusammenfassung und Fazit}
\label{sec:zusammenfassung}
%Abschließende Zusammenfassung bezüglich der angewendeten Methode, der
%erzielten experimentellen-analytischen Ergebnisse und dem Ergebnis aus der
%Diskussion und Einordnung.
%- Abschließende Reflektion zwischen angestrebten Versuchsziel und den erzielten
%Ergebnissen.
%- Gegebenenfalls: Ausblick -> experimentelle Verbesserungsvorschläge zur Erhöhung
%der Genauigkeit, weitere experimentelle Reihen zur „noch besseren“ Lösung oder
%Erweiterung des Wissenszuwachses aus der Aufgabenstellung

Für die Prüfung der Plausibilität der Messwerte wurden die Messwerte im Programm \textsc{VLE} die Messwerte in verschiedenen Darstellungen mit dem \textsc{UNIFAC}-Modell und dem \textsc{Wilson}-Modell verglichen. Für jede Darstellung und jedes Modell sind dabei Abweichungen zu verzeichnen, welche jedoch nicht die Haupttendenz der Messverläufe verfälschen. Annahmen, welche für die Auswertung der Messwerte getroffen wurden, können neben der Fehlerbetrachtung der Versuchsdurchführung Gründe dafür sein. Schlussendlich bewiesen die  genutzten $\text{G}^{\text{E}}$-Modelle eine gute Performance für das Fitting der Messwerte bzw. ließen sich die Messwerte mit diesen etablierten Modellen gut auf ihre Plausibilität prüfen und bestätigen. Eine wiederholte Messung, um festgestellte Abweichungen erklären zu könne wäre ratsam.\\
Die Zusammenhänge bzw. Abhängigkeiten der Systemparametern Temperatur $T$, Druck $p$, sowie der Anteile in der Flüssigphase $x^L$ und Gasphase $x^V$ ließen sich im Praktikumsversuch verständlich erfassen und auswerten.