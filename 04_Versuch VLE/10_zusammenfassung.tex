\section{Zusammenfassung und Fazit}
\label{sec:zusammenfassung}
%Abschließende Zusammenfassung bezüglich der angewendeten Methode, der
%erzielten experimentellen-analytischen Ergebnisse und dem Ergebnis aus der
%Diskussion und Einordnung.
%- Abschließende Reflektion zwischen angestrebten Versuchsziel und den erzielten
%Ergebnissen.
%- Gegebenenfalls: Ausblick -> experimentelle Verbesserungsvorschläge zur Erhöhung
%der Genauigkeit, weitere experimentelle Reihen zur „noch besseren“ Lösung oder
%Erweiterung des Wissenszuwachses aus der Aufgabenstellung

Im Versuch für die Charakterisierung der Oberflächenspannungen ließen sich die Einwirkung von Fremdstoffen, sowie die Temperaturabhängigkeit gut untersuchen. Die Messwerte scheinen im Vergleich zu Literaturwerten plausibel zu sein, weisen aber dennoch Abweichungen auf, welche durch ein genauere Arbeitsweise untersucht werden sollte.\\
Als Erkenntnis lässt sich feststellen, dass die Struktur der Fremdstoffe, die einer wässrigen Lösung zugegeben wird von massiver Bedeutung für den Einfluss auf Oberflächenspannungen ist. Somit könnten anhand bestimmter chemischer Strukturen Voraussagen getroffen, welche experimentell zu erwarten sind.
Natriumdodecylsulfat als anionisches Tensid bewies sich in diesem Versuch als sehr wirkungsvoll, wenn es um die Verringerung der Oberflächenspannung geht. 

Ebenfalls positiv auf die Oberflächenspannungverringerung wirkte sich die Temperaturerhöhung aus. Zwar wiesen für \SI{90}{\celsius} der Literaturwert und der Wert für die Regression Unterschiede auf, jedoch ist die allgemeine Tendenz der Temperaturabhängigkeit eindeutig erkennbar.

Möchte man also die Oberflächenspannung von wässrigen Flüssigkeiten für Phasenübergänge beeinflussen, so sollte für eine niedrige Oberflächenspannung die Temperatur hochgehalten und auf den Einsatz von Tensiden gesetzt werden.