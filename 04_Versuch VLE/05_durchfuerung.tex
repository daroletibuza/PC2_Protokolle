\newpage
\section{Versuchsdurchführung}
\label{sec:durchfuerung}
%Ausführliche Beschreibung der eigenen Versuchsdurchführung in Sätzen.
%- Einbindung der eigenen Beobachtungen und der Notizen aus dem
%Beobachtungsprotokoll.
%- Benennung der ggf. während des Versuches aufgetretenen Schwierigkeiten,
%Probleme oder unerwarteten Phänomene.

Für den Versuch zur Oberflächenspannung-Bestimmung wurde das \mbox{\textsc{Tensiometer K6} }der Firma \textsc{Krüss} verwendet. Mithilfe eines bereitgestellten Platin-Iridium-Ringes und mittels Thermostat konstant temperierten Messgefäß wurde die Oberflächenspannung rein mechanisch gemessen.\\
Vor dem Versuch ist das entsprechende Messgefäß mit Aceton zu spülen gewesen und Geometrie des Ringes zu prüfen. Zu beachten gilt es hierbei keine fasrigen Tücher oder vergleichbare Materialien zu nutzen, um Messfehler zu vermeiden. Ist das passiert ist noch auf die Ausrichtung des Messgerätes zu achten. Dies kann mit Hilfe von Stellschrauben in präzise, waagerechte Position gebracht werden.\linebreak
Nach Treffen dieser Vorbereitungen kann die Messung gestartet werden. Hierfür wird der Ring auf die Oberfläche der Probe abgesetzt und die Skala des Messgerätes mittel Stellschrauben und Handrad tariert. Der Probentisch wird nun soweit abgesenkt und die Kraft auf die Probe mittels Handrad zusätzlich erhöht bis die am Ring haftende Flüssigkeit abreißt. Es ist zu beachten, dass dabei im Kreuz der schwarz-weißen Markierung des Herstellers gearbeitet wird.\\
Dieses Vorgehen wird im ersten Versuchsteil für bidestilliertes und auf \SI{20}{\celsius} temperiertes Wasser für die Korrekturfaktorbestimmung angewendet.\linebreak
Im zweiten Versuchsteil erfolgt dann die Messung verschiedener selbst angesetzter Proben. In diesem Fall umfassen die Proben eine Kochsalzlösung (\SI{0,1}{\mol\per\liter}), Ethanol (\SI{0,1}{\mol\per\liter}) und eine Natriumdodecylsulfat-Lösung (\SI{0,001}{\mol\per\liter}).\linebreak
Im dritten Versuchsteil wird dann die Temperaturabhängigkeit der Oberflächenspannung, zwischen $20-\SI{60}{\celsius}$, mittels Thermostat und Wasser analysiert.