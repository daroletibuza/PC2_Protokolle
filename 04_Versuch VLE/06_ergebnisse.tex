\newpage
\section{Ergebnisse}
\label{sec:ergebnisse}
%5.) Inhalt des Kapitels: „Ergebnisse und Resultate
%- Darstellung der eigenen Versuchsergebnisse in Sätzen in Kombination mit
%entsprechenden Tabellen (u.a. aus Messprotokoll) und Abbildungen bzw. grafischer
%Darstellung der Ergebnisse.
%- Alle Abbildungen und Tabellen sind durchzunummerieren (Abb. 01, …, Tabelle 01: …)
%und mit einer Bild- bzw. Tabellenunterschrift zu versehen.
%- Der Inhalt von Abbildungen, z.B. der Verlauf eines Graphen, ist jeweils im Text zu
%erläutern und kurz zu beschreiben. Beschreibung der Verläufe von Funktionen bzw.
%der Graphen im Text unter Verweis auf die Abbildung bzw. Abbildungsnummer und
%Herausarbeitung ihrer „Kernaussage“ und Ursache bzw. Hintergründe besonderer
%„Verläufe“.
%- Rechenwege mit denen die experimentellen Daten für die Auswertung bearbeitet
%wurden bzw. der Gang der Auswertung muss vollständig beschrieben und für einen
%Leser nachvollziehbar sein.

\subsection*{Einwaagen der Stoffproben}
Zur Bestimmung der Oberflächenspannung der wässrigen Lösungen mussten diese erst hergestellt werden. Die Einwaagen dazu finden sich in Tabelle \ref{tab:einwaage}.
\vspace*{-4mm}
% Table generated by Excel2LaTeX from sheet 'Daten SURF'
\begin{table}[h!]
	\centering
	\caption{Einwaagen der Stoffproben}
	\begin{tabular}{l|ccc|c}
		\hline
		\textbf{Stoff} & $c$ in $\left[\si{\mol \per \liter}\right]$ & $\rho$ in $\left[\si{\gram \per \raiseto{3}\centi \meter}\right]$& $M$ in $\left[\si{\gram \per \mol}\right]$ & $m$ in $\left[\si{\gram}\right]$\\
		\hline
		Ethanol & 0,1   & 0,7995 & 46,07 & 0,461 \\
		NaCl  & 0,1   & 1,0025 & 58,44 & 0,584\\
		NaDDS & 0,001 & 1,0002 & 288,38 & 0,029 \\
		\hline
	\end{tabular}%
	\label{tab:einwaage}%
\end{table}%
\vspace*{-4mm}
\subsection*{Bestimmung des apparativen Korrekturfaktors}
Um die apparativen Abweichungen des Messgerätes zu erfassen, werden Kalibrierpunkte eines bekannten Stoffes, in diesem Fall Wasser bei \SI{20}{\celsius}, erfasst. Die Messwerte hierzu finden sich in Tab. \ref{tab:kalibrierung}
% Table generated by Excel2LaTeX from sheet 'Daten SURF'
\begin{table}[htbp]
	\centering
	\caption{Kalibrierpunkte für die Bestimmung des Korrekturfaktors $K_{\text{kal}}$}
	\begin{tabular}{c|c}
		\hline
		$T$ in $\left[\si{\celsius}\right]$ & $\sigma$ in $\left[\si{\milli \newton \per \meter}\right]$\\
		\hline
		\SI{20}{\celsius}& 69,8 \\
		\SI{20}{\celsius} & 69,8 \\
		\SI{20}{\celsius} & 69,9 \\
		\SI{20}{\celsius} & 69,9 \\
		\SI{20}{\celsius} & 70,0 \\
		\SI{20}{\celsius} & 70,0 \\
		\SI{20}{\celsius} & 70,0 \\
		\SI{20}{\celsius} & 70,0 \\
		\SI{20}{\celsius} & 70,0 \\
		\SI{20}{\celsius} & 70,0 \\
		\hline
		\textbf{Mittelwert} & 69,9 \\
		\hline
		\textbf{Standardabweichung} & 0,08\\
		\hline
		\textbf{relative Standardabweichung} & \SI{0,12}{\percent} \\
		\hline
		\textbf{Korrekturfaktor $K_{\text{kal}}$} & 1,041\\
		\bottomrule
	\end{tabular}%
	\label{tab:kalibrierung}%
\end{table}%
Mit dem berechneten, apparativen Korrekturfaktor lassen sich nun Messdaten korrigieren, die dem System Wasser entsprechen und keine zu hohen Abweichungen in der Dichte aufweisen.
\begin{flalign}
	K_{\text{kal}} &= \frac{\sigma^\ast}{\bar{\sigma}}\\
								&= \frac{\SI{72,8}{\milli \newton\per \meter}}{\SI{69,9}{\milli \newton \per \meter}}\\
								&= \underline{1,041}
\end{flalign}

\newpage

\subsection*{Bestimmung des stofflichen Korrekturfaktors und Korrektur der aufgenommenen Messwerte}

Um neben der Korrektur des Apparates auch die Korrektur für wässrige Lösungen mit Fremdstoffen einberechnen zu können, wird mithilfe einer Korrekturtabelle ein zusätzlicher Faktor für die zu untersuchenden Proben eingeführt (siehe Anhang, Abschnitt \ref{sec:anhang}).
Eine Berechnung der Oberflächenspannung folgt daher der Gleichung \eqref{gl:stoff_k}. Eine Beispielrechnung ist in Gleichung \eqref{gl:beispiel_k} erläutert und die Mess- und Rechenergebnisse in Tab. \ref{tab:Stoffdaten} zu finden.
\vspace*{-2.5mm}
\begin{flalign}
\label{gl:stoff_k}
	\sigma &= \sigma^\ast*K_{\text{kal}}*K
\end{flalign} 
\vspace*{-2.5mm}
\begin{table}[htbp]
	\centering
	\renewcommand*{\arraystretch}{1.2}
	\caption{Korrekturfaktoren der Stoffproben}
	\begin{tabular}{l|cc|c}
		\textbf{Stoff} & \textbf{Dichtefaktor} & \textbf{$\sigma$ gemessen} in $\left[\si{\milli \newton \per \meter}\right]$ & \textbf{K}\\
		\hline
		\textbf{NaCl} & 1,0   & 68,25 & 0,995 \\
		\textbf{Ethanol} & 0,8   & 64,42 & 1,006 \\
		\textbf{NaDDS} & 1,0   & 40,14 & 0,954 \\
	\end{tabular}%
	\label{tab:korrektur_ss}%
\end{table}%
\vspace*{-2.5mm}
\begin{flalign}
\label{gl:beispiel_k}
	\sigma &= \sigma^\ast*K_{\text{kal}}*K\\
	\sigma_{\text{Ethanol},1, \text{korr.}}				&= \SI{68,80}{\milli \newton \per \meter}*1,041*0,995\\
	&=\underline{\SI{71,22}{\milli \newton \per \meter}}
\end{flalign}

% Table generated by Excel2LaTeX from sheet 'Tabelle2'
\begin{table}[h!]
	\centering
	\renewcommand*{\arraystretch}{1.2}
	\caption{Oberflächenspannungen und statistische Kennwerte für die zu untersuchenden Proben bei \SI{20}{\celsius}}
	\resizebox*{15cm}{!}{
	\begin{tabular}{ccccccc}
		\hline
		\textbf{Nr.}& \textbf{NaCl} & \textbf{NaCl (korr.)}& \textbf{Ethanol}& \textbf{Ethanol (korr.)}& \textbf{NaDDS} & \textbf{NaDDS (korr.)}\\
		\hline
		\multicolumn{1}{c}{1} & \multicolumn{1}{c}{68,80} & 71,22 & \multicolumn{1}{c}{65,00} & 68,06 & \multicolumn{1}{c}{39,00} & 38,73 \\
		\multicolumn{1}{c}{2} & \multicolumn{1}{c}{68,00} & 70,39 & \multicolumn{1}{c}{64,50} & 67,54 & \multicolumn{1}{c}{40,00} & 39,72 \\
		\multicolumn{1}{c}{3} & \multicolumn{1}{c}{68,50} & 70,91 & \multicolumn{1}{c}{64,40} & 67,44 & \multicolumn{1}{c}{40,80} & 40,51 \\
		\multicolumn{1}{c}{4} & \multicolumn{1}{c}{68,10} & 70,49 & \multicolumn{1}{c}{63,50} & 66,49 & \multicolumn{1}{c}{41,00} & 40,71 \\
		\multicolumn{1}{c}{5} & \multicolumn{1}{c}{68,50} & 70,91 & \multicolumn{1}{c}{64,50} & 67,54 & \multicolumn{1}{c}{40,80} & 40,51 \\
		\multicolumn{1}{c}{6} & \multicolumn{1}{c}{68,00} & 70,39 & \multicolumn{1}{c}{64,60} & 67,65 & \multicolumn{1}{c}{41,00} & 40,71 \\
		\multicolumn{1}{c}{7} & \multicolumn{1}{c}{68,00} & 70,39 & \multicolumn{1}{c}{63,50} & 66,49 & \multicolumn{1}{c}{40,80} & 40,51 \\
		\multicolumn{1}{c}{8} & \multicolumn{1}{c}{68,50} & 70,91 & \multicolumn{1}{c}{64,60} & 67,65 & \multicolumn{1}{c}{39,00} & 38,73 \\
		\multicolumn{1}{c}{9} & \multicolumn{1}{c}{67,90} & 70,29 & \multicolumn{1}{c}{64,80} & 67,85 & \multicolumn{1}{c}{39,00} & 38,73 \\
		\multicolumn{1}{c}{10} & \multicolumn{1}{c}{68,20} & 70,60 & \multicolumn{1}{c}{64,80} & 67,85 & \multicolumn{1}{c}{40,00} & 39,72 \\
		\hline
		\textbf{Mittelwert in $\left[\si{\milli \newton \per \meter}\right]$} & 68,25& 70,65 & \multicolumn{1}{c}{64,42} & \multicolumn{1}{c}{67,46} & \multicolumn{1}{c}{40,14} & \multicolumn{1}{c}{39,86} \\
		\textbf{StaAbW in $\left[\si{\milli \newton \per \meter}\right]$} &       & 0,31  &       & 0,54  &       & 0,86 \\
		\textbf{Ausreißer+ in $\left[\si{\milli \newton \per \meter}\right]$} &       & 71,59 &       & 69,08 &       & 42,43 \\
		\textbf{Ausreißer- in $\left[\si{\milli \newton \per \meter}\right]$} &       & 69,71 &       & 65,84 &       & 37,28 \\
		\textbf{relat. StaAbW in $\left[\si{\percent}\right]$} &       & 0,44  &       & 0,80  &       & 2,15 \\
		\hline
   \end{tabular}}
	\label{tab:Stoffdaten}%
\end{table}%
\FloatBarrier
\newpage
\subsection*{Untersuchungen zur Temperaturabhängigkeit der Oberflächenspannung}
In Tabelle \ref{tab:temperatur_daten} sind die Oberflächenspannungen von bidestilliertem Wasser für verschiedene Temperaturen aufgetragen, sowie deren Korrektur mittels apparativen Korrekturfaktor $K_{\text{kal}}$.
\vspace*{-4mm}
% Table generated by Excel2LaTeX from sheet 'Tabelle3'
\begin{table}[h!]
	\renewcommand*{\arraystretch}{1.2}
	\centering
	\caption{Oberflächenspannungen von Wasser für verschiedene Temperaturen}
	\begin{tabular}{c|c|c}
		\hline
		$T$ in $\left[\si{\celsius}\right]$ & $\sigma$& $\sigma_{korr.}$ \\
		\hline
		20    & 69,8  & 72,7 \\
		20    & 69,8  & 72,7 \\
		20    & 69,9  & 72,8 \\
		20    & 69,9  & 72,8 \\
		20    & 70,0    & 72,9 \\
		20    & 70,0    & 72,9 \\
		20    & 70,0    & 72,9 \\
		20    & 70,0    & 72,9 \\
		20    & 70,0    & 72,9 \\
		20    & 70,0    & 72,9 \\
		30    & 69,0    & 71,8 \\
		30    & 69,0    & 71,8 \\
		30    & 69,0    & 71,8 \\
		35    & 68,2  & 71,0 \\
		35    & 68,4  & 71,2 \\
		35    & 68,5  & 71,3 \\
		40    & 67,8  & 70,6 \\
		40    & 67,8  & 70,6 \\
		40    & 68,0    & 70,8 \\
		49    & 66,0    & 68,7 \\
		49    & 66,5  & 69,2 \\
		49    & 67,0    & 69,7 \\
		\hline
	   \end{tabular}%
	\label{tab:temperatur_daten}%
\end{table}%
\FloatBarrier
\vspace*{-2mm}
Die daraus berechneten Mittelwerte und statistische Daten sind in Tab. \ref{tab:mittel_t} aufgezeigt und bildlich als Diagramm in Abb. \ref{dia:oberflaechenT} zu sehen.
\vspace*{-4mm}
% Table generated by Excel2LaTeX from sheet 'Tabelle3'
\begin{table}[h!]
	\centering
	\renewcommand*{\arraystretch}{1.2}
	\caption{Mittelwerte der temperaturabhängigen Oberflächenspannungen}
	\resizebox{16cm}{!}{
	\begin{tabular}{c|c|c|c|cc}
		$T$ in $\left[\si{\celsius}\right]$& $\sigma$ in $\left[\si{\milli \newton \per \meter}\right]$& StaAbW in $\left[\si{\milli \newton \per \meter}\right]$& rel. StaAbW in $\left[\si{\percent}\right]$& Ausreißer + in $\left[\si{\milli \newton \per \meter}\right]$& Ausreißer - in $\left[\si{\milli \newton \per \meter}\right]$\\
		\hline
		20    & 72,8  & 0,09& 0,12 & 73,1 & 72,5 \\
		30    & 71,8 & 0,00     & 0,00    & 71,8 & 71,8 \\
		35    & 71,2 & 0,16 & 0,22 & 71,6 & 70,7 \\
		40    & 70,6 & 0,12 & 0,17 & 71,00 & 70,3 \\
		49    & 69,2 & 0,52 & 0,75 & 70,8 & 67,7 \\
	\end{tabular}}
	\label{tab:mittel_t}%
\end{table}
\FloatBarrier


\begin{figure}[h!]
		\begin{center}
			\resizebox{0.8\textwidth}{!}{
				\begin{tikzpicture}[trim axis left, trim axis right]
				\begin{axis}[
				axis lines = left,
				width = 15cm,
				height = 11cm,
				xmin = 0,
				xmax = 55,
				ymin = 68,
				ymax = 76,
				%	ytick = {-4.5,-4,...,-1},
				%	xtick = {-10,-9,...,20},
				ylabel={Oberflächenspannung in \si{\milli \newton \per \meter}},
				%y label style={at={(0,0.5)}},
				xlabel={Temperatur in \si{\celsius}},
				legend style={at={(0.5,0.75)},anchor=west},
				%	y dir = reverse,
				]				
				\addplot [color=black, mark=*] coordinates{(20,72.8) (30,71.82156133829) (35,71.1623296158612) (40,70.6418835192069) (49,69.2193308550186) };
				
				\addplot +[mark=none, dashed, black, domain=0:56] {-0.122554665*x + 75.39392339};
				
				
			
				\legend{mittlere Oberflächen Spannungen, Regressionsgerade $\sigma=-\SI{0,123}{}*T + \SI{75.4}{} \, | \, R^2$ = \SI{0,988}{}}
				\end{axis}
				\end{tikzpicture}}
			\caption{Oberflächenspannung des Wassers in Abhängigkeit der Temperatur}
			\label{dia:oberflaechenT}
		\end{center}
	\end{figure}
\FloatBarrier
\vspace*{-4mm}
Im Graph des Diagramm in Abb. \ref{dia:oberflaechenT} lässt sich ein linearer Verlauf zwischen Oberflächenspannung $\sigma$ und Temperatur $T$ erkennen. Aufgrund eines Bestimmungsmaßes von $R^2=0,998$ sind die aufgenommen Messdaten als angemessen-genau vertretbar. Das zeigt sich auch in den relativen Standardabweichungen in den Tabellen Tab. \ref{tab:Stoffdaten} und Tab. \ref{tab:mittel_t}, welche jeweils unter \SI{1}{\percent} liegen.\\
Somit lässt sich über die Regressionsgerade auch die Oberflächenspannung bestimmen, welche bei \SI{90}{\celsius} theoretisch vorherrschen würde (siehe Gleichung \eqref{gl:neunzig})
\begin{flalign}
\label{gl:neunzig}
	\sigma &= f(T)\\
					&= \SI{-0,123}{\milli \newton \per \meter\per \celsius}*T\left[\si{\celsius}\right] + \SI{75,4}{\milli \newton \per \meter}\\
	\sigma(\SI{90}{\celsius}) &=\SI{-0,123}{\milli \newton \per \meter\per \celsius}*\SI{90}{\celsius} + \SI{75,4}{\milli \newton \per \meter}\\
	&=\underline{\SI{64,33}{\milli \newton \per \meter}}
\end{flalign}












