\newpage
\section{Ergebnisse}
\label{sec:ergebnisse}
%5.) Inhalt des Kapitels: „Ergebnisse und Resultate
%- Darstellung der eigenen Versuchsergebnisse in Sätzen in Kombination mit
%entsprechenden Tabellen (u.a. aus Messprotokoll) und Abbildungen bzw. grafischer
%Darstellung der Ergebnisse.
%- Alle Abbildungen und Tabellen sind durchzunummerieren (Abb. 01, …, Tabelle 01: …)
%und mit einer Bild- bzw. Tabellenunterschrift zu versehen.
%- Der Inhalt von Abbildungen, z.B. der Verlauf eines Graphen, ist jeweils im Text zu
%erläutern und kurz zu beschreiben. Beschreibung der Verläufe von Funktionen bzw.
%der Graphen im Text unter Verweis auf die Abbildung bzw. Abbildungsnummer und
%Herausarbeitung ihrer „Kernaussage“ und Ursache bzw. Hintergründe besonderer
%„Verläufe“.
%- Rechenwege mit denen die experimentellen Daten für die Auswertung bearbeitet
%wurden bzw. der Gang der Auswertung muss vollständig beschrieben und für einen
%Leser nachvollziehbar sein.

\subsection*{Tabellen der Messreihen 1 bis 4}
% Table generated by Excel2LaTeX from sheet 'Daten'
\begin{table}[h!]
	\centering
	\caption{Messwerte der Messreihe 1 für $T=\SI{303,15}{\kelvin}$}
	\resizebox{9.5cm}{!}{
	\begin{tabulary}{1.0\textwidth}{C|CCC|C|C|C}
		\textbf{Nr.} & \textbf{$p_0 \, \left[\si{\kilo \pascal}\right]$} & \textbf{$p \, \left[\si{\kilo \pascal}\right]$} & \textbf{$h \, \left[\si{\meter}\right]$} & \textbf{$V \left[\si{\milli \liter}\right]$} & \textbf{$n \, \left[\si{\kilo \mol}\right]$}&\textbf{$V_m \, \left[\si{\liter \per \kilo \mol }\right]$} \\
		\hline
		1     & 1726  & 1716 & 0,078 & 4,0   & 2,72E-06 & 0,87 \\
		2     & 1795  & 1783 & 0,087 & 3,8   & 2,68E-06 & 0,82 \\
		3     & 1876  & 1863 & 0,097 & 3,6   & 2,66E-06 & 0,78 \\
		4     & 1944  & 1923 & 0,109 & 3,4   & 2,60E-06 & 0,74 \\
		5     & 2036  & 2020 & 0,119 & 3,2   & 2,56E-06 & 0,69 \\
		6     & 2126  & 2109 & 0,129 & 3,0    & 2,51E-06 & 0,65 \\
		7     & 2213  & 2195 & 0,138 & 2,8   & 2,43E-06 & 0,61 \\
		8     & 2316  & 2296 & 0,149 & 2,6   & 2,36E-06 & 0,56 \\
		9     & 2427  & 2406 & 0,157 & 2,4   & 2,29E-06 & 0,52 \\
		10    & 2536  & 2513 & 0,170 & 2,2   & 2,19E-06 & 0,48 \\
		11    & 2643  & 2619 & 0,178 & 2     & 2,07E-06 & 0,43 \\
		12    & 2702  & 2677 & 0,190 & 1,8   & 1,91E-06 & 0,39 \\
		13    & 2707  & 2681 & 0,197 & 1,6   & 1,70E-06 & 0,35 \\
		14    & 2710  & 2682 & 0,210 & 1,4   & 1,48E-06 & 0,30 \\
		15    & 2713  & 2684 & 0,220 & 1,2   & 1,27E-06 & 0,26 \\
		16    & 2717  & 2686 & 0,231 & 1,0     & 1,06E-06 & 0,22 \\
		17    & 2730  & 2698 & 0,239 & 0,8   & 8,56E-07 & 0,17 \\
		18    & 2738  & 2705 & 0,245 & 0,7   & 7,51E-07 & 0,15 \\
		19    & 2755  & 2722 & 0,249 & 0,6   & 6,47E-07 & 0,13 \\
		20    & 2770  & 2736 & 0,255 & 0,5   & 5,42E-07 & 0,11 \\
		21    & 3517  & 3483 & 0,257 & 0,4   & 5,52E-07 & 0,09 \\
		22    & 4885  & 4850 & 0,262 & 0,35  & 6,73E-07 & 0,08 
	\label{tab:m1}%
\end{tabulary}}
\end{table}%
\FloatBarrier
\vspace*{-7mm}

% Table generated by Excel2LaTeX from sheet 'Daten'
\begin{table}[h!]
	\centering
	\caption{Messwerte der Messreihe 2 für $T=\SI{313,15}{\kelvin}$}
	\resizebox{9.5cm}{!}{
	\begin{tabulary}{1.0\textwidth}{C|CCC|C|C|C}
		\textbf{Nr.} & \textbf{$p_0 \, \left[\si{\kilo \pascal}\right]$} & \textbf{$p \, \left[\si{\kilo \pascal}\right]$} & \textbf{$h \, \left[\si{\meter}\right]$} & \textbf{$V \left[\si{\milli \liter}\right]$} & \textbf{$n \, \left[\si{\kilo \mol}\right]$}&\textbf{$V_m \, \left[\si{\liter \per \kilo \mol }\right]$} \\
		\hline
		1     & 1820  & 1810 & 0,077 & 4,0   & 2,87E-06 & 0,87 \\
		2     & 1896  & 1884 & 0,087 & 3,8   & 2,84E-06 & 0,83 \\
		3     & 1979  & 1966 & 0,099 & 3,6   & 2,80E-06 & 0,78 \\
		4     & 2064  & 2050 & 0,108 & 3,4   & 2,76E-06 & 0,74 \\
		5     & 2163  & 2147 & 0,118 & 3,2   & 2,72E-06 & 0,70 \\
		6     & 2260  & 2243 & 0,129 & 3     & 2,66E-06 & 0,65 \\
		7     & 2364  & 2345 & 0,14  & 2,8 	 & 2,60E-06 & 0,61 \\
		8     & 2480  & 2460 & 0,149 & 2,6   & 2,53E-06 & 0,57 \\
		9     & 2607  & 2586 & 0,159 & 2,4   & 2,46E-06 & 0,52 \\
		10    & 2740  & 2718 & 0,168 & 2,2   & 2,37E-06 & 0,48 \\
		11    & 2884  & 2860 & 0,18  & 2     & 2,26E-06 & 0,44 \\
		12    & 3027  & 3002 & 0,19  & 1,8   & 2,14E-06 & 0,39 \\
		13    & 3173  & 3147 & 0,198 & 1,6   & 1,99E-06 & 0,35 \\
		14    & 3313  & 3285 & 0,209 & 1,4   & 1,82E-06 & 0,30 \\
		15    & 3388  & 3359 & 0,219 & 1,2   & 1,59E-06 & 0,26 \\
		16    & 3396  & 3365 & 0,232 & 1     & 1,33E-06 & 0,22 \\
		17    & 3400  & 3368 & 0,239 & 0,8   & 1,06E-06 & 0,17 \\
		18    & 3408  & 3376 & 0,242 & 0,7   & 9,37E-07 & 0,15 \\
		19    & 3419  & 3386 & 0,251 & 0,6   & 8,05E-07 & 0,13 \\
		20    & 3451  & 3417 & 0,255 & 0,5   & 6,77E-07 & 0,11 \\
		21    & 4725  & 4690 & 0,26  & 0,41  & 7,62E-07 & 0,09 \\
		22    & 4945  & 4910 & 0,261 & 0,4   & 7,79E-07 & 0,09 
		\label{tab:m2}%
	\end{tabulary}}
\end{table}%
\FloatBarrier


% Table generated by Excel2LaTeX from sheet 'Daten'
\begin{table}[h!]
	\centering
	\caption{Messwerte der Messreihe 3 für $T=\SI{323,15}{\kelvin}$}
		\resizebox{10.5cm}{!}{
		\begin{tabulary}{1.0\textwidth}{C|CCC|C|C|C}
		\textbf{Nr.} & \textbf{$p_0 \, \left[\si{\kilo \pascal}\right]$} & \textbf{$p \, \left[\si{\kilo \pascal}\right]$} & \textbf{$h \, \left[\si{\meter}\right]$} & \textbf{$V \left[\si{\milli \liter}\right]$} & \textbf{$n \, \left[\si{\kilo \mol}\right]$}&\textbf{$V_m \, \left[\si{\liter \per \kilo \mol }\right]$} \\
		\hline
		1     & 1911  & 1901 & 0,077 & 4,0   & 3,01E-06 & 0,87 \\
		2     & 1995  & 1983 & 0,088 & 3,8   & 2,99E-06 & 0,83 \\
		3     & 2084  & 2070 & 0,099 & 3,6   & 2,95E-06 & 0,78 \\
		4     & 2177  & 2162 & 0,11  & 3,4   & 2,91E-06 & 0,74 \\
		5     & 2281  & 2265 & 0,118 & 3,2   & 2,87E-06 & 0,70 \\
		6     & 2390  & 2372 & 0,129 & 3     & 2,82E-06 & 0,65 \\
		7     & 2507  & 2488 & 0,139 & 2,8   & 2,76E-06 & 0,61 \\
		8     & 2639  & 2619 & 0,148 & 2,6   & 2,70E-06 & 0,57 \\
		9     & 2780  & 2758 & 0,16  & 2,4   & 2,62E-06 & 0,52 \\
		10    & 2934  & 2911 & 0,167 & 2,2   & 2,54E-06 & 0,48 \\
		11    & 3099  & 3075 & 0,178 & 2     & 2,44E-06 & 0,44 \\
		12    & 3286  & 3260 & 0,189 & 1,8   & 2,32E-06 & 0,39 \\
		13    & 3478  & 3451 & 0,201 & 1,6   & 2,19E-06 & 0,35 \\
		14    & 3667  & 3638 & 0,212 & 1,4   & 2,02E-06 & 0,30 \\
		15    & 3864  & 3834 & 0,222 & 1,2   & 1,82E-06 & 0,26 \\
		16    & 4028  & 3997 & 0,23  & 1     & 1,58E-06 & 0,22 \\
		17    & 4150  & 4118 & 0,239 & 0,8   & 1,30E-06 & 0,17 \\
		18    & 4302  & 4268 & 0,25  & 0,6   & 1,01E-06 & 0,13 \\
		19    & 4807  & 4772 & 0,256 & 0,5   & 9,46E-07 & 0,11 
		\label{tab:m3}%
	\end{tabulary}}
\end{table}%
\FloatBarrier

% Table generated by Excel2LaTeX from sheet 'Daten'
\begin{table}[h!]
	\centering
	\caption{Messwerte der Messreihe 4 für $T=\SI{328,15}{\kelvin}$}
		\resizebox{10.5cm}{!}{
		\begin{tabulary}{1.0\textwidth}{C|CCC|C|C|C}
		\textbf{Nr.} & \textbf{$p_0 \, \left[\si{\kilo \pascal}\right]$} & \textbf{$p \, \left[\si{\kilo \pascal}\right]$} & \textbf{$h \, \left[\si{\meter}\right]$} & \textbf{$V \left[\si{\milli \liter}\right]$} & \textbf{$n \, \left[\si{\kilo \mol}\right]$}&\textbf{$V_m \, \left[\si{\liter \per \kilo \mol }\right]$} \\
		\hline
		1     & 1961  & 1950 & 0,077 & 4,0   & 3,09E-06 & 0,87 \\
		2     & 2043  & 2031 & 0,087 & 3,8   & 3,06E-06 & 0,83 \\
		3     & 2136  & 2122 & 0,098 & 3,6   & 3,03E-06 & 0,78 \\
		4     & 2232  & 2217 & 0,108 & 3,4   & 2,99E-06 & 0,74 \\
		5     & 2343  & 2327 & 0,119 & 3,2   & 2,95E-06 & 0,70 \\
		6     & 2457  & 2439 & 0,13  & 3     & 2,90E-06 & 0,65 \\
		7     & 2580  & 2561 & 0,138 & 2,8   & 2,84E-06 & 0,61 \\
		8     & 2716  & 2696 & 0,148 & 2,6   & 2,78E-06 & 0,57 \\
		9     & 2870  & 2848 & 0,159 & 2,4   & 2,71E-06 & 0,52 \\
		10    & 3034  & 3011 & 0,17  & 2,2   & 2,62E-06 & 0,48 \\
		11    & 3206  & 3181 & 0,18  & 2     & 2,52E-06 & 0,44 \\
		12    & 3406  & 3380 & 0,189 & 1,8   & 2,41E-06 & 0,39 \\
		13    & 3621  & 3594 & 0,199 & 1,6   & 2,28E-06 & 0,35 \\
		14    & 3848  & 3819 & 0,211 & 1,4   & 2,12E-06 & 0,30 \\
		15    & 4079  & 4049 & 0,22  & 1,2   & 1,92E-06 & 0,26 \\
		16    & 4302  & 4271 & 0,229 & 1     & 1,69E-06 & 0,22 \\
		17    & 4401  & 4369 & 0,236 & 0,9   & 1,56E-06 & 0,20 \\
		18    & 4509  & 4476 & 0,244 & 0,8   & 1,42E-06 & 0,17 \\
		19    & 4627  & 4593 & 0,248 & 0,7   & 1,27E-06 & 0,15 \\
		20    & 4834  & 4800 & 0,251 & 0,6   & 1,14E-06 & 0,13 
		\label{tab:m4}%
	\end{tabulary}}
\end{table}%
\FloatBarrier

\begin{figure}[h!]
	\begin{center}
		\resizebox{0.8\textwidth}{!}{
			\begin{tikzpicture}[trim axis left, trim axis right]
			\begin{axis}[
			axis lines = left,
			width = 15cm,
			height = 11cm,
			xmin = 0,
			xmax = 4.5,
			ymin = 1500,
			ymax = 5000,
			%	ytick = {-4.5,-4,...,-1},
			%	xtick = {-10,-9,...,20},
			ylabel={Druck in \si{\kilo \pascal}},
			y label style={at={(-0.05,0.5)}},
			xlabel={Volumen in \si{\milli \liter}},
			legend style={at={(0.75,0.45)},anchor=west},
			%	y dir = reverse,
			]
			\addplot [color=black, mark=*] coordinates{(4,1715.593552) (3.8,1783.392808) (3.6,1863.058648) (3.4,1929.457656) (3.2,2020.123496) (3,2108.789336) (2.8,2194.588592) (2.6,2296.121016) (2.4,2406.053688) (2.2,2513.31928) (2,2619.251952) (1.8,2676.65096) (1.6,2680.717048) (1.4,2681.98264) (1.2,2683.64848) (0.999999999999999,2686.180904) (0.799999999999999,2698.113576) (0.699999999999999,2705.31308) (0.599999999999999,2721.779416) (0.499999999999999,2735.97892) (0.399999999999999,3482.712088) (0.35,4850.045008) };
			
			\addplot [color=brown, mark=*] coordinates{(4,1809.726968) (3.8,1884.392808) (3.6,1965.791816) (3.4,2049.591072) (3.2,2147.256912) (3,2242.789336) (2.8,2345.32176) (2.6,2460.121016) (2.4,2585.786856) (2.2,2717.586112) (2,2859.98512) (1.8,3001.65096) (1.6,3146.583632) (1.4,3285.116056) (1.2,3358.781896) (0.999999999999999,3365.047488) (0.799999999999999,3368.113576) (0.7,3375.713328) (0.6,3385.512584) (0.5,3416.97892) (0.41,4690.31184) (0.4,4910.178424)};
			
			\addplot [color=blue, mark=*] coordinates{(4,1900.726968) (3.8,1983.259392) (3.6,2070.791816) (3.4,2162.32424) (3.2,2265.256912) (3,2372.789336) (2.8,2488.455176) (2.6,2619.254432) (2.4,2758.65344) (2.2,2911.719528) (2,3075.251952) (1.8,3260.784376) (1.6,3451.183384) (1.4,3638.715808) (1.2,3834.381648) (0.999999999999999,3997.31432) (0.799999999999999,4118.113576) (0.6,4268.646) (0.5,4772.845504)};
			
			\addplot [color=red, mark=*] coordinates{(4,1950.726968) (3.8,2031.392808) (3.6,2122.925232) (3.4,2217.591072) (3.2,2327.123496) (3,2439.65592) (2.8,2561.588592) (2.6,2696.254432) (2.4,2848.786856) (2.2,3011.31928) (2,3181.98512) (1.8,3380.784376) (1.6,3594.450216) (1.4,3819.849224) (1.2,4049.64848) (0.999999999999999,4271.447736) (0.9,4369.513824) (0.8,4476.446496) (0.7,4593.912832) (0.6,4800.512584)};
			
			\addplot [dashed, black, mark=none] {3760};
			
			\legend{Messreihe 1: T=\SI{303,15}{\kelvin}, Messreihe 2: T=\SI{313,15}{\kelvin}, Messreihe 3: T=\SI{323,15}{\kelvin}, Messreihe 4: T=\SI{328,15}{\kelvin},Kritischer Druck $p_k=\SI{3760}{\kPa}$};
			\end{axis}
			\end{tikzpicture}
		}
		\caption{Isothermen der Messreihen 1 bis 4 von \ce{SF6}}
		\label{dia:isotherme}
	\end{center}
\end{figure}
\FloatBarrier

\begin{figure}[h!]
	\begin{center}
		\resizebox{0.8\textwidth}{!}{
			\begin{tikzpicture}[trim axis left, trim axis right]
			\begin{axis}[
			axis lines = left,
			width = 15cm,
			height = 11cm,
			xmin = 0,
			xmax = 3500,
			ymin = 0,
			ymax = 5.0e-6,
			%	ytick = {-4.5,-4,...,-1},
			%	xtick = {-10,-9,...,20},
			ylabel={Stoffmenge in \si{\kilo \mol}},
			%y label style={at={(0,0.5)}},
			xlabel={Druck in \si{\kilo \pascal}},
			legend style={at={(0.75,0.8)},anchor=west},
			%	y dir = reverse,
			]
			\addplot [color=blue, mark=*] coordinates{(1900.726968,2.82986389056993E-06) (1983.259392,2.80510379547671E-06) (2070.791816,2.77475575149913E-06) (2162.32424,2.73643762679967E-06) (2265.256912,2.69807030512243E-06) (2372.789336,2.64951422860374E-06) (2488.455176,2.59342488168714E-06) (2619.254432,2.53476005779796E-06) (2758.65344,2.46430357054391E-06)};
			
			\addplot +[mark=none, dashed, blue, domain=0:5000] {-3.85*(10^(-10))*x+3.56*(10^(-6))};
		
			
			\addplot [color=red, mark=*] coordinates{(1950.726968,2.86005279815834E-06) (2031.392808,2.82940475063201E-06) (2122.925232,2.80126871402144E-06) (2217.591072,2.76361761870505E-06) (2327.123496,2.7295243980632E-06) (2439.65592,2.68267094332018E-06) (2561.588592,2.62896608204171E-06) (2696.254432,2.56951875046436E-06) (2848.786856,2.50604444984312E-06)};
			
			\addplot +[mark=none, dashed, red, domain=0:4000] {-3.57e-10*x+3.56e-6};
		
			
			\legend{Messreihe 3: T=\SI{323,15}{\kelvin}, Regression Messreihe 3, Messreihe 4: T=\SI{328,15}{\kelvin}, Regression Messreihe 4}
			\end{axis}
			\end{tikzpicture}
		}
		\caption{Selektierte, berechnete Stoffmengen in Abhängigkeit vom Druck der überkritischen Messreihen 3 und 4 von \ce{SF6}}
		\label{dia:stoffmenge}
	\end{center}
\end{figure}
\FloatBarrier
\newpage

In Abb. \ref{dia:isotherme} sind die verschiedenen Isothermen der untersuchten Schwefelhexafluoridprobe dargestellt. Auffällig ist hierbei, dass alle Messreihen bei der Komprimierung linear bis leicht exponentiell ansteigen. Nach dieser Phase unterscheiden sich die Verhalten zwischen den ersten beiden und den letzten beiden Graphen der Messreihen. \linebreak 
Die Messreihen 1 und 2, welche geringere Temperaturen aufweisen, verlaufen nach der 1. Phase des Anstiegs bei der Komprimierung nun konstant weiter. Die Messreihen 3 und 4 jedoch steigen weiter exponentiell an. \linebreak 
Nach dem die Messreihen 1 und 2 während der Komprimierung eine konstante Phase beendet haben, folgt im letzten Schluss wiederum ein stark exponentieller Anstieg des Druckes.\\
Diese Zusammenhänge lassen sich mit den beobachteten Phasenwechseln der Messreihen 1 und 2, sowie der Betrachtung des kritischen Punktes von Hexafluorid \linebreak erklären.  
Der kritische Punkt beschreibt in diesem Versuch die den kritischen Druck $p_k$ eines realen Gases an welchem die Isotherme des Gases einen Sattelpunkt \mbox{besitzt \cite{Foth.2006}}. Ab diesem Punkt sind die flüssige und die gasförmige Phase eines Stoffes nicht mehr klar zu unterscheiden. Für Schwefelhexafluorid liegt dieser kritische Punkt bei \SI{37,6}{\bar} \cite{Wikipedia.2020b}.\\
Das lässt sich auch mit aufgenommenen Messwerten bestätigen, da die Graphen lediglich über dem kritischen Druck keine konstanten Abschnitte aufzeigen. Das ist auch erklärbar, das über diesem Druck faktisch kein Phasenwechsel mehr stattfindet.\linebreak 
Im Gegensatz zu den überkritischen Messreihen 3 und 4 weisen die unterkritischen Messreihen 1 und 2 jedoch diese konstanten Verläufe auf. Der Grund dafür lässt sich in den auftretenden Phasenwechsel dieser Messreihen erklären. So beschreiben diese Abschnitte des Verlaufes, dass jegliche Erhöhung des Gasdruckes zur Kondensation führt. Ein Teil dieser druckbestimmenden Teilchen wird durch diesen Vorgang jedoch der Gasphase entzogen. Der Druck im Messzylinder bleibt stabil. Verringert sich das Volumen im Messzylinder jedoch soweit, dass jegliche Gasteilchen kondensiert sind, so steigt der Druck im Messzylinder rapide, das Flüssigkeiten im Gegensatz zu Gasen nahezu inkompressibel sind.\\

In Abbildung \ref{dia:stoffmenge} ist Stoffmenge in Abhängigkeit des Druckes über einen linearen Geradenausgleich der Viralgleichung dargstellt. In der dargestellten Abbildung sind die Messreihen 3 und 4 mit ihren jeweligen Regressionen gezeigt. Es wurden hierfür Messwerte aussortiert, welche das Bestimmtheitsmaß massiv verändert habe, sprich große Abweichungen zum linearen Zusammenhang aufwiesen. Die Bestimmtheitsmaße weisen mit $R^2 \approx 0,9984$ und $R^2 \approx 0,9963$ annehmbare Abweichungen auf. Über den Achsenabnitt $a$ wird auf diese Weise die mittlere Stoffmenge beider Isothermen bestimmt.

\newpage
\subsection*{Regressionsgleichungen der überkritischen Messreihen 3 und 4}
\textbf{Messreihe 3:}
\begin{flalign}
	n(p)_3 &=\SI{-3,84E-10}{}*p+\SI{3,56E-6}{} \, \, |\, R^2=0,9984
\end{flalign}
\textbf{Messreihe 4:}
\begin{flalign}
	n(p)_4 &= \SI{-3,58E-10}{}*p+\SI{3,56E-6}{}  \, \, |\, R^2=0,9963
\end{flalign}

\subsection*{Berechnung des korrigierten Druck $p$ aus $p_0$}
\begin{flalign}
p	&= p_0-h*g*\rho_{\tiny{\ce{Hg}}}\\
	&= \SI{1820}{\kilo \pascal}-\SI{0,077}{\meter}*\SI{9,81}{\meter\per \raiseto{2}\second}*\SI{13,6}{\kg\per \liter}\\
	&= \underline{\SI{1810}{\kilo \pascal}}
\end{flalign}
\subsection*{Berechnung der Stoffmenge $n$}
\begin{flalign}
p*V	&= n*R*T \\
n	&= \frac{p*V}{R*T}\\
	&= \frac{\SI{1900}{\kilo \pascal}*\SI{4}{\milli \liter}}{\SI{8,314}{\joule \per \mole \per \kelvin}*\SI{323,15}{\kelvin}}\\
	&= \underline{\SI{2,83E-06}{\kmol}	}
\end{flalign}

\subsection*{Berechnung des molaren Volumens $V_m$}
\begin{flalign}
V_m &= \frac{V}{\overline{n}}\\
	&= \frac{\SI{4}{\milli \liter}}{\SI{3,56E-06}{\kmol}}\\
	&= \underline{\SI{1,12}{\liter \per\kmol}}
\end{flalign}

% Table generated by Excel2LaTeX from sheet 'Daten'
\begin{table}[h!]
	\centering
	\caption{Achsenabschnitt a bzw. Stoffmenge n und Anstieg b bestimmt aus den überkritischen Messreihen 3 und 4}
	\resizebox{11cm}{!}{
		\begin{tabulary}{\textwidth}{C|CC|C}
			\textbf{Kennwert}&\textbf{Messreihe 3} & \textbf{Messreihe 4} & \textbf{Mittelwert}\\
			\hline
			a&\SI{3,56E-06}{\kmol}&\SI{3,56E-06}{\kmol}&\SI{3,56E-06}{\kmol}\\
			b&\SI{-3,84E-10}{\kmol \per \kilo \pascal}&\SI{-3,58E-10}{\kmol \per \kilo \pascal}&-\\
			\hline		
			\label{tab:Vkonstanten}%
	\end{tabulary}}
\end{table}%
\FloatBarrier


