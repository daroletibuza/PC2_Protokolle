\newpage
\section{Versuchsdurchführung}
\label{sec:durchfuerung}
%Ausführliche Beschreibung der eigenen Versuchsdurchführung in Sätzen.
%- Einbindung der eigenen Beobachtungen und der Notizen aus dem
%Beobachtungsprotokoll.
%- Benennung der ggf. während des Versuches aufgetretenen Schwierigkeiten,
%Probleme oder unerwarteten Phänomene.

Am Arbeitsplatz zur Untersuchung des p-V-T-Verhaltens von Schwefelhexafluorid ist ein druckfester Messzylinder über einer Auffangwanne aufgestellt. Im Messzylinder selbst befindet sich der gasförmige, zu untersuchende Stoff. Am unteren Ende des Zylinders ist eine Quecksilbersäule zuerkennen, welche mittels Handrad reguliert werden kann um das Volumen im Messzylinder einzustellen. An der Apparatur ist zu dem ein Thermostat anschlossen, welches die isothermen Betriebsbedingungen durch umströmen des Messzylinders mit Wasser, sicherstellen soll. Zur Überprüfung der konstanten Temperatur ist zusätzlich ein Flüssigkeitsthermometer angebracht worden. Zur Messung des Drucks ist unterhalb der Quecksilbersäule ein Manometer befestigt.\\
Für die eigentliche Versuchsdurchführung wird nun für verschiedene, über das Thermostat eingestellte und über das Thermometer überprüfte Temperaturen, der Druck im Messzylinder über verschiedene Volumina handschriftlich aufgenommen. Dabei war es zu beachten, dass bei der Durchführung ein Druck von über \SI{50}{\bar} zu vermeiden ist. Neben den aufgenommen Drücken werden ebenfalls die jeweiligen Volumina, sowie die Höhe der Quecksilbersäule zur Korrigierung des gemessenen Drucks durch das Manometer notiert. Die Verringerung des Volumens erfolgte während des Versuches hauptsächlich in \SI{0,2}{ml}-Schritten und zum Ende hin jedoch in \SI{0,1}{ml}-Schritten. Die Messreihen 1 bis 4 wurden für die Temperaturen \SI{303,15}{\kelvin}, \SI{313,15}{\kelvin}, \SI{323,15}{\kelvin} und \SI{328,15}{\kelvin} durchgeführt.\\

Im Versuch ließ sich beobachten, dass mit Verringerung des Volumens der Druck im Messzylinder steigt. Jedoch steigt der Druck nicht einfach linear an, sonder verhält sich in verschiedenen Phasen der Volumenverringerung und je nach Temperatur unterschiedlich. Rein optisch sind für die Temperaturen der Messreihen 1 und 2 Phasenwechsel in vom gasförmigen in den flüssigen Aggregatzustand wahrzunehmen, welche ein Teil der Erklärung der aufgenommenen Druckdaten sein können.\linebreak
Weitere Ausführungen dazu sind unter dem folgendem Abschnitt \ref{sec:ergebnisse} aufgeführt.