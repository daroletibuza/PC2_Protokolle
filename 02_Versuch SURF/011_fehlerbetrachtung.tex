\newpage
\section{Fehlerbetrachtung}
\label{sec:fehler}

Anhand der sehr sensiblen Größe der Oberflächenspannung können sich schnell systematische Fehler in der Arbeitsweise dieses Versuches auftun, welche nicht in den Messergebnissen wieder zu finden sind. So gilt es zu beachten, dass kein Bruchteil von Fasern durch, beispielsweise Papiertücher bei Trocknen, in das Messgefäß gelangen. Diese können bereits das Messergebnis verfälschen. Stattdessen sollte auf das flüchtige Aceton als Lösungsmittel zurückgegriffen werden, um Wasser-, Ethanol- oder Tensidrückstände zuverlässig zu entfernen.\\
Des Weiteren geben die statistischen Kennwerte in den Tabellen \ref{tab:kalibrierung}, \ref{tab:Stoffdaten} und \ref{tab:mittel_t} Aussagen über die Präzision der Messdaten. Im Vergleich mit dem Ausreißerkriterium ist kein Messwert dieser Messreihen auszuschließen. Weiterhin spricht die relative Standardabweichung von weniger als \SI{1}{\percent} für die Präzision der Messwerte.\linebreak
Dennoch ist es nicht zu verachten, dass durch Arbeitsweise oder fehlerhafte Kalibrierung nicht dennoch Abweichungen gegenüber Literaturwerten vorkommen. Möchte man diese Unsicherheiten vermeiden, so sollte die in diesem Versuch überprüften Zusammenhänge und Messdaten nochmals mit einem anderen Messverfahren durchgeführt werden.