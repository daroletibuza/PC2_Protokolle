\section{Zusammenfassung und Fazit}
\label{sec:zusammenfassung}
%Abschließende Zusammenfassung bezüglich der angewendeten Methode, der
%erzielten experimentellen-analytischen Ergebnisse und dem Ergebnis aus der
%Diskussion und Einordnung.
%- Abschließende Reflektion zwischen angestrebten Versuchsziel und den erzielten
%Ergebnissen.
%- Gegebenenfalls: Ausblick -> experimentelle Verbesserungsvorschläge zur Erhöhung
%der Genauigkeit, weitere experimentelle Reihen zur „noch besseren“ Lösung oder
%Erweiterung des Wissenszuwachses aus der Aufgabenstellung

Im Versuch ließen sich die Literaturwerte, sowie die Modelle für diese Auswertung gut auf Plausibilität und Funktionalität prüfen. Da die Messergebnisse den Erwartungen im guten Maße entsprachen, ist kein Grund zur Annahme, dass die Messgeräte grundlegende Fehler in ihrer Funktionalität aufweisen. Zwar weichen \textsc{Antoine}-Parameter und die molare Verdampfungsenthalpie dezent von den Literaturwerten ab, zeigen jedoch aufgrund der geringen Abweichungen die gute Präzision  des Modells auf. Um noch genauere Ergebnisse zu erhalten könnte andere Software geprüft werden, welche womöglich andere Berechnungsmethoden nutzt. Auch eine feinere Abstufung für die Aufnahme der Messwerte könnte genauere Ergebnisse zur Folge haben. \\
Schlussendlich lässt sich sagen, dass sich das $p$-$T$-Verhalten für Isopropanol, anhand der aufgebauten Ebulliometrie-Apparatur, gut beschreiben ließ und daher keine Verbesserungen anzumerken sind.