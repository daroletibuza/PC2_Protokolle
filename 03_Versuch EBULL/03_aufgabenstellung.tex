\section{Einleitung und Versuchsziel}
\label{sec:aufgabenstellung}
%- Darstellung des Versuchsziels und der in dem Versuch bearbeiteten Fragestellungen.
%- Kurze Einführung grundlegenden theoretischen Zusammenhänge und der
%Gleichungen die für den Versuch und die Versuchsauswertung relevant sind.
%- Alle Abbildungen sind durchzunummerieren (Abb. 01, …) und mit einer Bild- bzw.
%Tabellenunterschrift zu versehen. Externe Quellen sind in der Bildunterschrift als
%Literatur-Nummer (Quelle: [1]) oder Literatur-Kürzel (Quelle: [Schmidt2015])
%anzugeben.Physikalische Chemie
%FB Ingenieur- und Naturwissenschaften
%Protokollvorlage PC-II Praktikum, SoSe 2020 2
%- Alle Formeln sind durchzunummerieren.
%- Benennung der experimentellen Geräte und Hilfsmittel mit denen der Versuch
%durchgeführt wird (Bsp: Thermostat Firma/Typ XY, Druckmessröhre Firma/Typ Z).

Im Praktikumsversuch "`Untersuchungen zur Dampfdruckkurve einer reinen Flüssigkeit mittels Ebulliometrie"' werden mittels einer Siedeapparatur die Siedepunkte von reinem Isopropanol unter verschiedenen Drücken gemessen. Ziel ist es mit den aufgenommenen Messwerten Dampfdruckkurven zu erstellen und diese mit Literaturwerten bzw. dem Programm \textsc{ZUST} abzugleichen.

\subsection*{Zusammenhang zwischen Dampfdruck und Siedetemperatur}
Der Dampfdruck ist essenziell für die Beschreibung von Siedevorgängen. Er beschreibt denjenigen Druck, welcher sich bei einem im Gleichgewicht befindenden Dampf, innerhalb eines geschlossenen Behälters,  einstellt. Steht dieser Dampf mit einer flüssigen oder festen Phase in Wechselwirkung, so kann er auch als ein Bestreben der Teilchen in die Gasphase überzugehen, beschrieben werden. Der Dampfdruck ist lediglich von der Temperatur abhängig. Mit steigender Temperatur steigt auch der Dampfdruck im Behälter. \linebreak
Am Siedepunkt einer Flüssigkeit erreicht der Dampfdruck den Umgebungsdrucks des Systems. Für Standardbedingungen wäre der Dampfdruck einer Flüssigkeit am Siedepunkt mit \SI{101,3}{\kilo \pascal} zu bemessen. An diesem Punkt ist der Dampfdruck so groß, dass ausreichend Teilchen der flüssigen Phase den Druck der umgebenen Phase überwinden können und in die Gasphase übergehen.  
Besonders häufig wird für die mathematische Beschreibung dieses Zusammenhanges die aus empirischen Parametern zusammengesetzte \textsc{Antoine}-Gleichung (Gl. \ref{gl:antoine}) genutzt.\\
Im Praktikum werden diese Grundlagen der Ebulliometrie genutzt, um die ausgewählte Dampfkurven-Modelle, nach \textsc{Antoine} und \textsc{Frost-Kalkwarf}, mit experimentell bestimmten Messungen zu vergleichen und in Bezug auf ihre Funktionalität auszuwerten.\\

\textbf{\textsc{Antoine}-Gleichung:}
\begin{flalign}
\label{gl:antoine}
	\ln(p)	&= A- \frac{B}{C+T}
\end{flalign}

\textbf{molare Verdampfungsenthalpie}
\begin{flalign}
\label{gl:verdamp}
\Delta^{\text{LV}} H_m &= -R* \frac{\ln\left(\frac{p_{s,T2}}{p_{s,T1}}\right)}{\frac{1}{T_2}-\frac{1}{T_1}}
\end{flalign}


\textbf{\textsc{Frost-Kalkwarf}-Gleichung:}
\begin{flalign}
\label{gl:fkw}
\ln(p) &= A+\frac{B}{T} + C* \ln(T) + \frac{D*p}{T^2}
\end{flalign}