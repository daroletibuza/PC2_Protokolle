\newpage
\section{Fehlerbetrachtung}
\label{sec:fehler}

Fehlerquellen in diesem Versuch lassen sich, hauptsächlich auf den Versuchsaufbau selbst zurück führen. So ist die Aufrechterhaltung des jeweiligen Druckes in der Apparatur mit Schwankungen verknüpft. Die Verzögerungen bis der Sollwert wie eingestellt ist, führt zu Abweichungen im Endergebnis. \\
Weiterhin spritzt die siedende Flüssigkeit nur quasi-stationär gegen den Temperatursensor. Zwar lässt sich auf der Anzeige die Konstanz der Temperatur ablesen, jedoch sind Abweichungen, auch durch den Sensor selbst, nicht auszuschließen.\\
Da zuletzt die reine Isopropanol-Probe schon in der Apparatur vorlegen hat, ist die Reinheit nicht überprüft worden. Mögliche Verunreinigungen der Probe könne ebenfalls zu Abweichungen der Siedetemperatur führen.
