\section{Fehlerbetrachtung}
\label{sec:fehler}

Mögliche Fehlerquellen bei diesem Versuch beschränken sich hauptsächlich auf die Messeinrichtungen und das Thermostat am Versuchsstand. Da der Druck elektrisch gemessen wurde ist es möglich, dass das Messgerät sensitiv auf Temperaturen reagiert. Auch das Thermostat könnte Schwankungen in der Temperaturregelung unterliegen, welche selbst über die analoge Temperaturmessung nicht erfasst wurden. \linebreak
Diese möglichen Fehlerquellen sind vermutlich jedoch minimal in Bezug auf die Ablesefehler, welche durch simple Parallaxenfehler beim Ablesen des Volumens oder beim Verrutschen des Lineals zur Messung der Quecksilbersäule. Abhilfe für die möglichen Messfehler mit Lineal könnte ein angeklebtes Maßband schaffen oder eventuell die Fixierung des Lineals mit Kabelbindern.
