\section{Einleitung und Versuchsziel}
\label{sec:aufgabenstellung}
%- Darstellung des Versuchsziels und der in dem Versuch bearbeiteten Fragestellungen.
%- Kurze Einführung grundlegenden theoretischen Zusammenhänge und der
%Gleichungen die für den Versuch und die Versuchsauswertung relevant sind.
%- Alle Abbildungen sind durchzunummerieren (Abb. 01, …) und mit einer Bild- bzw.
%Tabellenunterschrift zu versehen. Externe Quellen sind in der Bildunterschrift als
%Literatur-Nummer (Quelle: [1]) oder Literatur-Kürzel (Quelle: [Schmidt2015])
%anzugeben.Physikalische Chemie
%FB Ingenieur- und Naturwissenschaften
%Protokollvorlage PC-II Praktikum, SoSe 2020 2
%- Alle Formeln sind durchzunummerieren.
%- Benennung der experimentellen Geräte und Hilfsmittel mit denen der Versuch
%durchgeführt wird (Bsp: Thermostat Firma/Typ XY, Druckmessröhre Firma/Typ Z).

Im Praktikumsversuch "`p-V-T-Verhalten eines Reinstoffs"' wird das Verhalten des reinen Probegases Schwefelhexafluorid \ce{SF6} unter isothermen Bedingungen untersucht. Das Gas wird über eine Volumenverkleinerung und eine Druckmessung  in Messwerten charakterisiert. Diese ermöglichen Berechnungen der Stoffmenge in der Gasphase, sowie dessen molaren Volumina. Zudem werden aus den ermittelten Daten für Schwefelhexafluorid Isothermen eines Zustandsdiagramms dargestellt.\\

\textbf{Stoffmenge aus der idealen Gasgleichung:}
\begin{flalign}
	p*V	&= n*R*T \\
	n	&= \frac{p*V}{R*T}
\end{flalign}

\textbf{Molares Volumen:}
\begin{flalign}
	V_m &= \frac{\overline{n}}{V}
\end{flalign}

\textbf{Van-der-Waals-Konstanten für Binnendruckparameter a und Ko-Volumen b:}
\begin{flalign}
	a	&= n(p=0)
\end{flalign}
\begin{flalign}
		b	&= f'(p,n)
\end{flalign}

\textbf{Van-der-Waals-Gleichung (intensiv):}
\begin{flalign}
	\left(p+\frac{a}{{V_m}^2}\right)*(V_m-b) &= R*T
\end{flalign}

\textbf{Druckkorrektur durch die \ce{Hg}-Säule:}
\begin{flalign}
	p	&= p_0-h*g*\rho_{\tiny{\ce{Hg}}}
\end{flalign}