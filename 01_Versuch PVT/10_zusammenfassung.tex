\section{Zusammenfassung und Fazit}
\label{sec:zusammenfassung}
%Abschließende Zusammenfassung bezüglich der angewendeten Methode, der
%erzielten experimentellen-analytischen Ergebnisse und dem Ergebnis aus der
%Diskussion und Einordnung.
%- Abschließende Reflektion zwischen angestrebten Versuchsziel und den erzielten
%Ergebnissen.
%- Gegebenenfalls: Ausblick -> experimentelle Verbesserungsvorschläge zur Erhöhung
%der Genauigkeit, weitere experimentelle Reihen zur „noch besseren“ Lösung oder
%Erweiterung des Wissenszuwachses aus der Aufgabenstellung

Zusammenfassend lässt sich sagen, dass die aufgenommenen Messwerte durch die \textsc{Peng-Robinson}-Graphen bestätigen lassen konnten und lediglich die Berechnung der Virialkoeffizienten stärkere Abweichungen zu den Literaturwerten aufweist. Die Messdaten geben jedoch Aufschluss über das Grundverständnis in Bezug auf Zustandsdiagrammen, sowie deren besondere Punkte und Abschnitte. Eine weitere Betrachtung mit Berechnungsmodellen könnte angebracht sein. Weitere Literatur für die Virialkoeffizienten der Messreihen 3 und 4 könnten ebenfalls angebracht sein und in eine Neubewertung einfließen.