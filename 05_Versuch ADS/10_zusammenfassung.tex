\section{Zusammenfassung und Fazit}
\label{sec:zusammenfassung}
%Abschließende Zusammenfassung bezüglich der angewendeten Methode, der
%erzielten experimentellen-analytischen Ergebnisse und dem Ergebnis aus der
%Diskussion und Einordnung.
%- Abschließende Reflektion zwischen angestrebten Versuchsziel und den erzielten
%Ergebnissen.
%- Gegebenenfalls: Ausblick -> experimentelle Verbesserungsvorschläge zur Erhöhung
%der Genauigkeit, weitere experimentelle Reihen zur „noch besseren“ Lösung oder
%Erweiterung des Wissenszuwachses aus der Aufgabenstellung

Zusammenfassend lässt sich sagen, dass in diesem theoretischen Versuch für die Adsorption von Iod an Aktivkohle, das \textsc{Freundlich}-Modell für die Beschreibung der Adsorptionsisotherme vorzuziehen ist. Zwar lässt das Modell nach \textsc{Langmuir} durchaus eine Beschreibung der Messwerte zu, jedoch ist diese unzureichend für qualitative Aussagen.\\

Es ist möglich, dass die getroffenen Annahmen für das \textsc{Langmuir}-Modell an dieser Stelle einfach nicht zutreffend sind. Die Beschreibung nach \textsc{Freundlich} ist für die genutzte Aktivkohle und Beladungen zwischen \SI{20}{\percent} und \SI{80}{\percent} als ausreichend bis gut zu bewerten. Die Kombination von \textsc{Freundlich} und \textsc{Langmuir} könnte womöglich ein besseres Adsorptionsisothermen-Modell hervorbringen, wenn es möglich ist die Annahmen beider Modelle zu kombinieren und mathematisch zusammenzufassen 