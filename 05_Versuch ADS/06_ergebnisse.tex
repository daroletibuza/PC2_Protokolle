\newpage
\section{Ergebnisse}
\label{sec:ergebnisse}
%5.) Inhalt des Kapitels: „Ergebnisse und Resultate
%- Darstellung der eigenen Versuchsergebnisse in Sätzen in Kombination mit
%entsprechenden Tabellen (u.a. aus Messprotokoll) und Abbildungen bzw. grafischer
%Darstellung der Ergebnisse.
%- Alle Abbildungen und Tabellen sind durchzunummerieren (Abb. 01, …, Tabelle 01: …)
%und mit einer Bild- bzw. Tabellenunterschrift zu versehen.
%- Der Inhalt von Abbildungen, z.B. der Verlauf eines Graphen, ist jeweils im Text zu
%erläutern und kurz zu beschreiben. Beschreibung der Verläufe von Funktionen bzw.
%der Graphen im Text unter Verweis auf die Abbildung bzw. Abbildungsnummer und
%Herausarbeitung ihrer „Kernaussage“ und Ursache bzw. Hintergründe besonderer
%„Verläufe“.
%- Rechenwege mit denen die experimentellen Daten für die Auswertung bearbeitet
%wurden bzw. der Gang der Auswertung muss vollständig beschrieben und für einen
%Leser nachvollziehbar sein.

\subsection*{Kalibriergerade}
% Table generated by Excel2LaTeX from sheet 'Daten'
\begin{table}[h!]
	\renewcommand*{\arraystretch}{1.2}
	\centering
	%\rowcolors{2}{white}{gray!25}
	\caption{Messwerte für die Kalibriergerade zur Bestimmung der Iod-Konzentration mittels Fotometrie}
	\label{tab:kalibirerung}
%	\resizebox{10.5cm}{!}{
		\begin{tabulary}{1.0\textwidth}{C|C}
			\hline
			\textbf{Konzentration $\left[\si{\milli \gram \per \liter}\right]$} & \textbf{Extinktion $\left[-\right]$}\\
			\hline
			 \rowcolor[rgb]{ .776,  .937,  .808} 2     & 0,3770 \\
			\rowcolor[rgb]{ .776,  .937,  .808} 4     & 0,5740 \\
			\rowcolor[rgb]{ .776,  .937,  .808} 6     & 0,7970 \\
			\rowcolor[rgb]{ .776,  .937,  .808} 8     & 0,9895 \\
			\rowcolor[rgb]{ .776,  .937,  .808} 10    & 1,2015 \\
			\rowcolor[rgb]{ .776,  .937,  .808} 12    & 1,3925 \\
			\hline
			\rowcolor[rgb]{ 1,  .78,  .808} 16    & 1,8815 \\
			\rowcolor[rgb]{ 1,  .78,  .808} 20    & 2,5880 \\
			\hline			
	\end{tabulary}
%}
\end{table}%
\FloatBarrier

 \definecolor{newred}{rgb}{1,  .78,  .808}
 \definecolor{newgreen}{rgb}{.800,  .937,  .808}
 
\textcolor{newgreen} {Extinktion unter 1,5 - innerhalb des Kalibrierbereiches}\\
\textcolor{newred} {Extinktion über 1,5 - außerhalb des Kalibrierbereiches}

\begin{figure}[h!]
	\begin{center}
		\resizebox{0.8\textwidth}{!}{
			\begin{tikzpicture}[trim axis left, trim axis right]
			\begin{axis}[
			axis lines = left,
			width = 15cm,
			height = 11cm,
			xmin = 0,
			xmax = 25,
			ymin = 0,
			ymax = 3,
			%	ytick = {-4.5,-4,...,-1},
			%	xtick = {-10,-9,...,20},
			ylabel={Extinktion},
			%y label style={at={(0,0.5)}},
			xlabel={Konzentration in \si{\milli \gram \per \liter}},
			legend style={at={(0.4,0.25)},anchor=west},
			%	y dir = reverse,
			]	
			
			\addplot [color=green!75!black, mark=*] coordinates{(2,0.377) (4,0.574) (6,0.797) (8,0.9895) (10,1.2015) (12,1.3925) };
			
			\addplot [color=red, mark=*, only marks] coordinates{(16,1.8815) (20,2.588)};
			
			\addplot +[mark=none, solid, black, domain=0:25] {0.1021786 *x+0.1733333};
			
			\addplot +[mark=none, dashed, red, domain=0:25] {0*x+1.5};
			
			
			
			\legend{Messwerte mit $E<\SI{1,5}{}$, Messwerte mit $E>\SI{1,5}{}$, Kalibriergerade $|\, E = \SI{0,10217}{}*c+\SI{0,17333}{} \,|\, R^2 = \SI{0,9996}{}$}
			\end{axis}
			\end{tikzpicture}
		}
		\caption{Kalibriergerade zur fotometrischen Bestimmung der Iod-Konzentration}
		\label{dia:kalibrierung}
	\end{center}
\end{figure}
\FloatBarrier

\subsection*{Messwerte}


% Table generated by Excel2LaTeX from sheet 'Daten'
\begin{table}[h!]
	\renewcommand*{\arraystretch}{1.2}
	\centering
	\rowcolors{2}{gray!25}{white}
	\caption{Einwaagen an Aktivkohle und Extinktionswerte für \SI{20}{\milli \liter} Iod-Stammlösung}
	\label{tab:messwerte}
	\resizebox{0.8\textwidth}{!}{
		\begin{tabulary}{1.2\textwidth}{CC|C|C}
 \hline
\textbf{Sollmasse Aktivkohle $\left[\si{\gram}\right]$}&\textbf{Istmasse Aktivkohle $\left[\si{\gram}\right]$}&\textbf{Verdünnung}&\textbf{Extinktion $\left[-\right]$}\\
 \hline
 0,1   & 0,10705 & 1:500 & 1,287 \\
 0,2   & 0,19927 & 1:200 & 1,396 \\
 0,3   & 0,30557 & 1:100 & 0,786 \\
 0,4   & 0,40078 & 1:25  & 0,963 \\
 0,5   & 0,50025 & 1:10  & 0,953 \\
 0,6   & 0,60068 & 1:10  & 0,587 \\
 0,7   & 0,70079 & 1:10  & 0,373 \\
 0,8   & 0,8035 & -     & 1,329 \\
 0,9   & 0,90485 & -     & 0,886 \\
 1,0   & 1,00149 & -     & 0,554 \\
 \hline
	\end{tabulary}
}
\end{table}%
\FloatBarrier

% Table generated by Excel2LaTeX from sheet 'Daten'
\begin{table}[h!]
	\renewcommand*{\arraystretch}{1.2}
	\centering
	\rowcolors{2}{gray!25}{white}
	\caption{Berechnete GGW-Konzentrationen, Massen und Beladung für die jeweilige Masse an Aktivkohle}
	\label{tab:ggw}
	\resizebox{0.9 \textwidth}{!}{
		\begin{tabulary}{1.3\textwidth}{CC|CC|C}
			\hline
			\textbf{GGW-Konzentration (o. Verdünnung) $\left[\si{\milli \gram \per \liter}\right]$} &\textbf{GGW-Konzentration (m. Verdünnung) $\left[\si{\milli \gram \per \liter}\right]$} & \textbf{Masse an ad. Iod $\left[\si{\milli \gram}\right]$} & \textbf{Masse an nicht-ad. Iod $\left[\si{\milli \gram}\right]$} & \textbf{Beladung $\left[-\right]$}\\
			\hline
			10,90 & 5449,61 & 91,008 & 108,992 & 0,85 \\
			11,97 & 2393,20 & 152,136 & 47,864 & 0,76 \\
			6,00  & 599,60 & 188,008 & 11,992 & 0,62 \\
			7,73  & 193,21 & 196,136 & 3,864 & 0,49 \\
			7,63  & 76,30 & 198,474 & 1,526 & 0,40 \\
			4,05  & 40,48 & 199,190 & 0,810 & 0,33 \\
			1,95  & 19,54 & 199,609 & 0,391 & 0,28 \\
			11,31 & 11,31 & 199,774 & 0,226 & 0,25 \\
			6,97  & 6,97  & 199,861 & 0,139 & 0,22 \\
			3,73  & 3,73  & 199,925 & 0,075 & 0,20 \\
			\hline
\end{tabulary}}
\end{table}%
\FloatBarrier

% Table generated by Excel2LaTeX from sheet 'Daten'
\begin{table}[h!]
	\renewcommand*{\arraystretch}{1.2}
	\centering
	\rowcolors{2}{gray!25}{white}
	\caption{gefittete Messwerte der Adsorptionsisothermen-Modelle nach \textsc{Freundlich} und \textsc{Langmuir} }
	\label{tab:isothermen}
	\resizebox{0.7\textwidth}{!}{
		\begin{tabulary}{1.0\textwidth}{CCCC|CCCCC}
			\hline
			\multicolumn{4}{c|}{\textbf{\textsc{Freundlich}}}&\multicolumn{5}{c}{\textbf{\textsc{Langmuir}}}\\
			\hline
			$\ln\left(b\right)$&$\ln\left(c\right)$&$c$&$b_\text{fit}$&$\frac{1}{b}$&$\frac{1}{c}$&$\frac{1}{b_\text{fit}}$&$c$&$b_\text{fit}$\\
			\hline
			-0,162 & 8,603 & 5449,61 & 0,83  & 1,176 & 0,000 & 1,391 & 5449,61 & 0,72 \\
			-0,270 & 7,780 & 2393,20 & 0,71  & 1,310 & 0,000 & 1,403 & 2393,20 & 0,71 \\
			-0,486 & 6,396 & 599,60 & 0,54  & 1,625 & 0,002 & 1,468 & 599,60 & 0,68 \\
			-0,715 & 5,264 & 193,21 & 0,43  & 2,043 & 0,005 & 1,651 & 193,21 & 0,61 \\
			-0,924 & 4,335 & 76,30 & 0,36  & 2,520 & 0,013 & 2,064 & 76,30 & 0,48 \\
			-1,104 & 3,701 & 40,48 & 0,32  & 3,016 & 0,025 & 2,667 & 40,48 & 0,37 \\
			-1,256 & 2,973 & 19,54 & 0,28  & 3,511 & 0,051 & 4,046 & 19,54 & 0,25 \\
			-1,392 & 2,426 & 11,31 & 0,25  & 4,022 & 0,088 & 5,985 & 11,31 & 0,17 \\
			-1,510 & 1,942 & 6,97  & 0,23  & 4,527 & 0,143 & 8,847 & 6,97  & 0,11 \\
			-1,611 & 1,315 & 3,73  & 0,20  & 5,009 & 0,268 & 15,358 & 3,73  & 0,07 \\
			\hline			
	\end{tabulary}
}
\end{table}%
\FloatBarrier

\subsection*{Grafische Bestimmung der jeweiligen Modellparameter}

\begin{figure}[h!]
	\begin{center}
		\resizebox{0.6\textwidth}{!}{
			\begin{tikzpicture}[trim axis left, trim axis right]
			\begin{axis}[
			axis lines = left,
			width = 15cm,
			height = 11cm,
			xmin = 0,
			xmax = 10,
			ymin = -1.9,
			ymax =0,
			%	ytick = {-4.5,-4,...,-1},
			%	xtick = {-10,-9,...,20},
			ylabel={$\ln\left(b\right)$},
			%y label style={at={(0,0.5)}},
			xlabel={$\ln\left(c\right)$},
			legend style={at={(0.5,0.4)},anchor=west},
			%	y dir = reverse,
			]
			
			%Freundlich
			\addplot [color=black, mark=*] coordinates{(8.60329926923593,-0.162350730567179) (7.78038492838138,-0.269885286128077) (6.39626921736071,-0.485694781302059) (5.26376475748397,-0.714605117268654) (4.33472958803521,-0.92445029985661) (3.70092360665564,-1.10380166539752) (2.97251241956648,-1.25584691060573) (2.42571067414201,-1.39179148517026) (1.94229182004197,-1.51014953265123) (1.31520211867294,-1.61129942328996)  };
			
			%Gerade
			\addplot +[color=black, mark=none, dashed, domain=0:10]{0.196*x-1.871};
			
			\legend{Messpunkte, Regressionsgerade $|\,\ln\left(b\right) = n*\ln\left(c\right)+\ln\left(k\right) \, | \, R^2=\SI{0,791}{}$}
			\end{axis}
			\end{tikzpicture}
		}
		\caption{Natürlich logarithmierte Beladung in Abhängigkeit der natürlich logarithmierten Konzentration zur Bestimmung der \textsc{Freundlich}-Parameter}
		\label{dia:freundlich}
	\end{center}
\end{figure}
\FloatBarrier

\begin{figure}[h!]
	\begin{center}
		\resizebox{0.6\textwidth}{!}{
			\begin{tikzpicture}[trim axis left, trim axis right]
			\begin{axis}[
			axis lines = left,
			width = 15cm,
			height = 11cm,
			xmin = 0,
			xmax = 0.3,
			ymin = 0,
			ymax =6,
			ytick = {0,0.5,...,6},
				xtick = {0,0.05,...,0.3},
			ylabel={$\frac{1}{b}$ in $\left[-\right]$},
			%y label style={at={(0,0.5)}},
			xlabel={$\frac{1}{c}$ in $\left[\si{\liter \per \milli \gram}\right]$},
			legend style={at={(0.4,0.3)},anchor=west},
			%	y dir = reverse,
			]
			
			%Langmuir
			\addplot [color=black, mark=*] coordinates{(1.83499379997434E-04,1.17627272378124) (4.17851300825674E-04,1.30981418825778) (1.66776775998756E-03,1.62530384737419) (5.17578242778749E-03,2.04337962759455) (1.31054174555671E-02,2.52048236996079) (2.47007021986877E-02,3.01560859459922) (5.11745766754114E-02,3.51081045823202) (8.84152622687379E-02,4.02204904384225) (0.14337498329547,4.52740773898272) (0.268420065048787,5.00931622271394) };
			
			%Regression
			\addplot +[color=black, mark=none, dashed, domain=0:0.3]{52.0718*x+1.3811};
			
			%Fit
			
			\legend{Messpunkte, Regressionsgerade $|\,\frac{1}{b} =\SI{ 52,072}{}*\frac{1}{c}+\SI{1,381}{} \, | \, R^2 =\SI{0,385}{} $}
			\end{axis}
			\end{tikzpicture}
		}
		\caption{Reziproke Beladung in Abhängigkeit der reziproken Konzentration zur Bestimmung der \textsc{Langmuir}-Parameter}
		\label{dia:langmuir}
	\end{center}
\end{figure}
\FloatBarrier

\newpage
\subsection*{Darstellung der modellierten Adsorptionsisotherme}

\begin{figure}[h!]
	\begin{center}
		\resizebox{0.8\textwidth}{!}{
			\begin{tikzpicture}[trim axis left, trim axis right]
			\begin{axis}[
			axis lines = left,
			width = 15cm,
			height = 11cm,
			xmin = 0,
			xmax = 6000,
			ymin = 0,
			ymax = 1,
			%	ytick = {-4.5,-4,...,-1},
			%	xtick = {-10,-9,...,20},
			ylabel={Beladung in $\left[\si{\kilogram \per \kilogram}\right]$},
			%y label style={at={(0,0.5)}},
			xlabel={GGW-Konzentration in $\left[\si{\milli \gram \per \liter}\right]$ },
			legend style={at={(0.5,0.5)},anchor=west},
			%	y dir = reverse,
			]
			
			%Messwerte
			\addplot [color=black, mark=*, only marks] coordinates{(5449.60969358033,0.850142981115304) (2393.19585226611,0.763467069577347) (599.603868111383,0.615269570434834) (193.207503204008,0.489385323459054) (76.3043224979611,0.396749452373895) (40.4846790166608,0.331608021608289) (19.5409530467203,0.284834516672706) (11.3102644762903,0.24862948937209) (6.97471746475591,0.220876947174344) (3.7255039030642,0.199628044136176)};
			
			%Freundlich
				\addplot [color=red, mark=x] coordinates{(5449.60969358034,0.830640714709619) (2393.19585226611,0.706968332243285) (599.603868111383,0.539063463184041) (193.207503204008,0.43180416568539) (76.3043224979611,0.359952216284985) (40.4846790166608,0.317922209869718) (19.5409530467203,0.275643596956376) (11.3102644762903,0.247643089818697) (6.97471746475591,0.225266760162947) (3.7255039030642,0.199225326967278) };
			
			%Langmuir
			\addplot [color=blue, mark=x] coordinates{(5449.60969358033,0.719060453529748) (2393.19585226611,0.712805735802859) (599.603868111383,0.681202537439819) (193.207503204008,0.605817974756826) (76.3043224979611,0.48459679676595) (40.4846790166608,0.374902724202981) (19.5409530467203,0.247163738282417) (11.3102644762903,0.167081849935808) (6.97471746475591,0.113033404205869) (3.7255039030642,6.51115213977936E-02) };
			
			
			\legend{Messwerte, \textsc{Freundlich}-Isotherme $|\, b=\SI{0,154}{}*c^{\SI{0,196}{}}$,\textsc{Langmuir}-Isotherme $|\, b=\left(\SI{52,072}{}*c+\SI{1,381}{}\right)^{-1}$}
			\end{axis}
			\end{tikzpicture}
				}
		\caption{Messwerte und auf Messwerten gefittete Adsorptionsisotherme nach \textsc{Freundlich} und \textsc{Langmuir}}
		\label{dia:isotherme}
	\end{center}
\end{figure}
\FloatBarrier