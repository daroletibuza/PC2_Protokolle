\section{Diskussion der Ergebnisse}
\label{sec:diskussion}
%7.) Inhalt des Kapitels: „Diskussion der Ergebnisse
%- Die Ergebnisse des Versuches und der in der Versuchsanleitung geforde
%Bestimmung einzelner Parameter soll unter wissenschaftlichen Aspekten disku
%werden. Dazu ist u.a. folgenden Fragen nachzugehen: Was sagen die Ergebn
%aus? Welche Annahmen und Näherungen wurden für die Ergebnisfindung gema
%Sind die Ergebnisse plausibel und decken sich mit Literatur- bzw. ande
%Vergleichswerten? Was sind mögliche Fehlerquellen und wie groß wird ihr Einf
%abgeschätzt?
%- Vergleich und Einordnung der eigenen Ergebnisse zu Literaturwerten
%- Abschluss des Kapitels: Finale Wertung der erarbeiteten Ergebnisse und Darstellung
%ihrer Kernaussage in kurzer prägnanter Form.

Vergleicht man die beiden Modelle für die Beschreibung von Adsorptionsisothermen, so sollte einem bewusst sein, dass beide Modelle unterschiedliche Annahmen und Vereinfachungen treffen. Dementsprechend fallen die Ergebnisse beider Modelle für diesen theoretischen Versuch auch recht unterschiedlich aus. \\

Bereits bei der Bestimmung der Parameter wird klar, dass das Modell nach \textsc{Freundlich} für die Messwerte mit den gegebenen mathematischen Umformungen, welche das Modell benötigt,  gut erfasst sind. Erkennbar ist dies in Abb. \ref{dia:freundlich} und des Bestimmtheitsmaßes von 0,992.\\
Im Vergleich dazu wird im Modell nach \textsc{Langmuir} eine lineare Regression auf einen kurvenförmigen Verlauf angewendet. Ein Unterschied, wenn bedacht wird, dass die Messwerte nach mathematischer Umformung im \textsc{Freundlich}-Modell einen linearen Verlauf aufweisen. Des Weiteren zeigt sich im Bestimmtheitsmaß ebenfalls eine geringe Qualität der Regression im Vergleich zum \textsc{Freundlich}-Modell. Grund hierfür ist jedoch unter anderem, dass das Fitting für das \textsc{Langmuir}-Modell basierend auf der endgültigen Adsorptionsisotherme berechnet wurde. Danach ergeben sich für diesen Versuch, dass im Endergebnis bessere Werte nach dem \textsc{Langmuir}-Modell berechnet werden können, obwohl ein qualitativ schlechtere Regression für die Bestimmung der Parameter in Abb. \ref{dia:langmuir} durchgeführt wurde.\\

Der entscheidende Vergleich dieser Diskussion zeigt sich nach den vorangegangenen Ausführungen dennoch in Abb. \ref{dia:isotherme}. Das Fitting der berechneten Isothermen ist eindeutig dargestellt und zeigt im Bestimmtheitsmaß der Modelle, dass die \textsc{Freundlich}-Isotherme mit $R^2=0,988$ eine deutlich bessere Beschreibung dieses Versuches darstellt als das \textsc{Langmuir}-Modell mit $R^2=0,651$.