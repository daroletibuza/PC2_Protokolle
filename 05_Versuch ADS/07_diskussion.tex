\section{Diskussion der Ergebnisse}
\label{sec:diskussion}
%7.) Inhalt des Kapitels: „Diskussion der Ergebnisse
%- Die Ergebnisse des Versuches und der in der Versuchsanleitung geforde
%Bestimmung einzelner Parameter soll unter wissenschaftlichen Aspekten disku
%werden. Dazu ist u.a. folgenden Fragen nachzugehen: Was sagen die Ergebn
%aus? Welche Annahmen und Näherungen wurden für die Ergebnisfindung gema
%Sind die Ergebnisse plausibel und decken sich mit Literatur- bzw. ande
%Vergleichswerten? Was sind mögliche Fehlerquellen und wie groß wird ihr Einf
%abgeschätzt?
%- Vergleich und Einordnung der eigenen Ergebnisse zu Literaturwerten
%- Abschluss des Kapitels: Finale Wertung der erarbeiteten Ergebnisse und Darstellung
%ihrer Kernaussage in kurzer prägnanter Form.

Die Messwerte zeigten im Vergleich zu den $\text{G}^\text{E}$-Modellen ein gutes Fitting, welches für die Plausibilität der Messdaten spricht. Auch ein Vergleich des azeotropen Punktes aus den Messwerten,verglichen mit dem \textsc{UNIFAC}-Modell, spricht für diese These. Die Werte hierfür sind dem Protokolldeckblatt zu entnehmen.\\
Für weitere Plausibilitätsprüfungen könnten die Messwerte auf weitere $\text{G}^\text{E}$-Modelle, wie zum Beispiel das \textsc{NRTL}-Modell, überprüft werden und die berechneten azeotropen Punkte verglichen werden.