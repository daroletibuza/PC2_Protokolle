\section{Einleitung und Versuchsziel}
\label{sec:aufgabenstellung}
%- Darstellung des Versuchsziels und der in dem Versuch bearbeiteten Fragestellungen.
%- Kurze Einführung grundlegenden theoretischen Zusammenhänge und der
%Gleichungen die für den Versuch und die Versuchsauswertung relevant sind.
%- Alle Abbildungen sind durchzunummerieren (Abb. 01, …) und mit einer Bild- bzw.
%Tabellenunterschrift zu versehen. Externe Quellen sind in der Bildunterschrift als
%Literatur-Nummer (Quelle: [1]) oder Literatur-Kürzel (Quelle: [Schmidt2015])
%anzugeben.Physikalische Chemie
%FB Ingenieur- und Naturwissenschaften
%Protokollvorlage PC-II Praktikum, SoSe 2020 2
%- Alle Formeln sind durchzunummerieren.
%- Benennung der experimentellen Geräte und Hilfsmittel mit denen der Versuch
%durchgeführt wird (Bsp: Thermostat Firma/Typ XY, Druckmessröhre Firma/Typ Z).

Im theoretischen Praktikumsversuch "`Adsorptionsisotherme"' werden die zur Verfügung gestellten Messdaten, für die Adsorption einer wässrigen Kaliumiodid-Lösung   an Aktivkohle, ausgewertet. Es ist anzunehmen, dass die Messdaten unter Raumtemperatur aufgenommen wurden. Zur Beschreibung der Adsorptionsisotherme sind die Modelle von \textsc{Freundlich} und \textsc{Langmuir} anzuwenden.

\section*{Theoretische Grundlagen}
Um das sich einstellende Gleichgewicht der Beladung $b$ bei konstanter Temperatur beschreiben zu können, werden Adsorptionsisotherme genutzt. Sie geben als Funktion die Beladung $b$, je nach Aggregatzustand des Adsorptives, in Abhängigkeit vom Druck $p$ oder der Konzentration $c$ an. Alternativ kann auch anstelle der Beladung $b$ der Bedeckungsgrad $\Theta$ genutzt werden.

\subsection*{Grundbegriffe}


\subsection*{Freundlich-Isotherme}
\begin{flalign}
	\tag*{| $\ln\left(...\right)$}	b &= k*c^n \\
	\ln\left(b\right) &= \ln\left(k\right)+n*\ln\left(c\right)
\end{flalign}


\subsection*{Langmuir-Isotherme}
\begin{flalign}
	\tag*{| $*\,c$ \hspace*{8.5mm}} 
	\frac{1}{b_{\infty}} *\frac{1}{K}*\frac{1}{c} + \frac{1}{b_{\infty}} &= \frac{1}{b}\\
	\tag*{| $*\,b$\hspace*{10mm}} \frac{1}{b_{\infty}} *\frac{1}{K}+ \frac{1}{b_{\infty}}*c&= \frac{c}{b}\\
	\tag*{| $\frac{b}{b_{\infty}}=\Theta$} \frac{b}{b_{\infty}} *\frac{1}{K}+ \frac{b}{b_{\infty}}*c&= c\\
	\Theta *\frac{1}{K} + \Theta*c &= c\\
	\Theta *\left(\frac{1}{K}+c\right) &= c\\
	\Theta = \frac{b}{b_{\infty}} &= \frac{c}{\frac{1}{K}+c} 
\end{flalign}

KI Lösung warum ? Citavi Website
GEsetzmäßigkeit ? Lambert BEersches Gesetz
