\section{Fehlerbetrachtung}
\label{sec:fehler}

Als Fehlerquellen lassen sich hauptsächlich manuelle Arbeitsschritte, welche durch den Praktikumsteilnehmer durchgeführt wurden feststellen. So sind das Entnehmen der Proben mittels Spritze und das manuelle zu dosieren der Komponenten Messfehlern unterlegen. Diese wirken sich bis auf die Berechnung der \textsc{Wilson}-Parameter aus. Hinzukommt, dass die beschriebene Vorgehensweise im Praktikum zu ersten Mal für den Teilnehmer durchgeführt wurde und so auch kleinste, handwerkliche Fehler in die erst aufgenommenen Messwerte einfließen, welche zu Abweichungen führen können. Eine weitere Fehlerquelle stellt die Bestimmung der Brechungsindices mittels Refraktometer dar. Zwar erfolgten die Messungen in der Regel oft ohne Probleme, jedoch ist in seltenen Fällen eine Unschärfe aufgetreten, welche jedoch meist mittels wiederholter Messung auszuschließen waren. Dennoch ist es zu empfehlen eine Dreifachbestimmung der jeweiligen Messwerte durchzuführen. Auch die schwankende Temperaturanzeige in der dritten Nachkommastelle kann zu beeinflussen der Messwerte führen. Diese Fehler sind lediglich nur in der Darstellung des Siedediagramms zu verzeichnen. Betrachtet an zuletzt die Kalibrierfunktion mit einem Bestimmtheitsmaß von \mbox{$R^2=0,9996$ }so könnten auch hierdurch minimale Fehler eine Auswirkung auf die Mess- und Auswertungsergebnisse haben.