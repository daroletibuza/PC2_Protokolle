\vspace*{-10mm}
\section{Fehlerbetrachtung}
\label{sec:fehler}
\textit{Da der Versuch selbst nicht durchgeführt wurde sind mögliche Fehlerquellen lediglich aus reiner Überlegung an dieser Stellen dargelegt.} \\

Da Messungen jeglicher Art zufälligen Fehlern unterliegen, müsste auch in diesem Versuch von einer Messtoleranz ausgegangen werden. Diese sind jedoch bei der Messung der Extinktion mit Werten unter 1,5 als sehr gering anzunehmen. Auch das Einwiegen der Proben kann als sehr genau angenommen werden. Vorrangig lassen sich Fehlerquellen im Pipettieren und Ansetzen der Stammlösung, sowie Verunreinigungen der Gefäße vermuten. Auch die Einhaltung der Grenzen an der Schüttelmaschine lassen mögliche Fehler zu, ebenso wie die Wartezeit bis zu chemischen Gleichgewicht. \\
Fehler im weiteren Sinne der Auswertung lassen sich aufgrund von Annahmen und Vereinfachungen der Modelle nicht vermeiden. Eine Darstellung des realen Zustandes ist nur bedingt möglich und erfordert je nach Probe eine fachmännische Einschätzung. Welches Modell sich am Besten für die Beschreibung der Adsorptionsisotherme eignet wird unter der Diskussion im Abschnitt \ref{sec:diskussion} beschrieben.